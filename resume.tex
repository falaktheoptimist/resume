%%%%%%%%%%%%%%%%%%%%%%%%%%%%%%%%%%%%%%%%%
% Medium Length Professional CV
% LaTeX Template
% Version 2.0 (8/5/13)
%
% This template has been downloaded from:
% http://www.LaTeXTemplates.com
%
% Original author:
% Trey Hunner (http://www.treyhunner.com/)
%
% Important note:
% This template requires the resume.cls file to be in the same directory as the
% .tex file. The resume.cls file provides the resume style used for structuring the
% document.
%
%%%%%%%%%%%%%%%%%%%%%%%%%%%%%%%%%%%%%%%%%

%----------------------------------------------------------------------------------------
%	PACKAGES AND OTHER DOCUMENT CONFIGURATIONS
%----------------------------------------------------------------------------------------

\documentclass{resume} % Use the custom resume.cls style
\usepackage[left=0.45in,top=0.6in,right=0.45in,bottom=0.6in]{geometry} % Document margins
\usepackage{color}
\usepackage{hyperref}
\newcommand{\tab}[1]{\hspace{.2667\textwidth}\rlap{#1}}
\newcommand{\itab}[1]{\hspace{0em}\rlap{#1}}
\name{Falak Shah} % Your name

%\address{A-163,Ashoknagar Society, Satellite, Ahmedabad-15} % Your address
%%\address{123 Pleasant Lane \\ City, State 12345} % Your secondary addess (optional)
%\address{(+91)940.813.0235 \\ falaktheoptimist@gmail.com} % Your phone number and email

\begin{document}
\noindent	falaktheoptimist@gmail.com \hfill \hfill  A-163,Ashoknagar Society, \\
\noindent (+91) 940-813-0235 \hfill \hfill  Satellite, Ahmedabad-380015, \\
\noindent \href{https://www.linkedin.com/in/falak-shah/}{LinkedIn} \hfill \hfill Gujarat, India.\\

	%----------------------------------------------------------------------------------------
	%	EDUCATION SECTION
	%----------------------------------------------------------------------------------------
	
	\begin{rSection}{Education}
		
		{\bf Dhirubhai Ambani Institute of Information and Communication Technology}  
		\\ M.Tech, ICT \hfill { CPI : 9.96/10}
		\\ Recipient of President's Gold Medal for Academic Excellence  
		
			{\bf Nirma University}  
			\\ B.Tech, Electronics and Communication \hfill { CPI : 8.31/10}
	 
		%Minor in Linguistics \smallskip \\
		%Member of Eta Kappa Nu \\
		%Member of Upsilon Pi Epsilon \\
		
		
	\end{rSection}
	%----------------------------------------------------------------------------------------
	%	TECHNICAL STRENGTHS SECTION
	%----------------------------------------------------------------------------------------
	
	\begin{rSection}{Technical Strengths}
		
		\begin{tabular}{ @{} >{\bfseries}l @{\hspace{4ex}} l }
			Languages &  Python, Matlab, C/C++, C\# \\
			DL Toolkits &  TensorFlow, Keras, Scikit Learn\\
			Software \& Technologies &   Apache Spark, Simulink, LaTeX, MS Office, SVN, Git\\
			Areas of Interest &   Machine Learning, Computer Vision, Signal Processing, Convex Optimization
		\end{tabular}
		
	\end{rSection}
	
	%----------------------------------------------------------------------------------------
	%	WORK EXPERIENCE SECTION
	%----------------------------------------------------------------------------------------
	
	\begin{rSection}{Experience}
		
		\begin{rSubsection}{Research Scientist, Infocusp}{June 2015- Present}{}{}
			\item Developed a library for visual representation of feature learning  in neural networks: \href{https://github.com/InFoCusp/tf_cnnvis}{tf\_cnnvis}.
			\item Working with our client, Cerebellum Capital (CCI) on employing Machine learning to discover and trading strategies likely to perform well out of sample
			\item Designed a portfolio optimizer for CCI which involved translating their financial domain constraints and concepts into a convex optimization problem
			\item Keeping up-to-date with the advancements in the fields of Machine Learning and Computer vision.

		\end{rSubsection}
		
		
		%------------------------------------------------
		
		\begin{rSubsection}{Graduate Teaching Assistant, DA-IICT}{July, 2013 - May, 2015}{}{}
			\item Conducted labs/ tutorial sessions for the B.Tech batch at DA-IICT for the following subjects:
			\item Digital Signal Processing 
			\item Signal and Systems
			\item Analog and Digital Communication
		\end{rSubsection}
		
			\begin{rSubsection}{Intern, Indian Space Research Organization}{Jan, 2013 - May, 2013}{}{}
				\item Designed and ran simulations on a highly robust GPS receiver model using Simulink. It made use of Kalman filter for seamless tracking loop operation even under severe ionospheric scintillations. 
			\end{rSubsection}
	\end{rSection}
	
	
	%	EXAMPLE SECTION
	%----------------------------------------------------------------------------------------
	
	\begin{rSection}{Achievements} \itemsep -2pt
		\item President's Gold Medal for Academic Excellence, 2015 Masters batch at DA-IICT
		\item Stood first in "Pixel Pundit" - a National level image processing competition organized by IISc Bengaluru and Carl Zeiss
		\item Stood first in circuit debugging contest at a National level Techfest at Nirma
		University.
		\item Ranked in National Top 0.25\% in Graduate Aptitude Test in Engineering (GATE)
%		\item Ranked in National Top 1\% in All India Engineering Entrance Examination (AIEEE)

	\end{rSection}
	
	%----------------------------------------------------------------------------------------

	\begin{rSection}{Publications}
		\item Co-authoring a book "Professional TensorFlow" for Packt Publishing (expected completion: Mar, 2018). This will serve as a student as well as instructor guide for practical training on TensorFlow. 
		\item F. Shah, P. Shah and R. Dubey, "Specularity Removal for Robust Road Detection," IEEE International Conference Region 10 (TENCON), Singapore, 2016.
		\item F. Shah, K. Shah and D. Shah, "BER performance improvement with combination of OVSF spreading and convolution code in inter-satellite links using FSO," Annual IEEE India Conference (INDICON), Mumbai, 2013.
	\end{rSection}
		

	\vspace{0.3 in}
		\begin{rSection}{Notable Projects}
			
			\begin{rSubsection}{tfcnnvis}{Jan, 2017 - Jun, 2017}{}{}{}
				\item Designed an open source library for visualizing the learning happening in convolutional neural networks: \href{https://github.com/InFoCusp/tf_cnnvis}{tf\_cnnvis}. We studied the existing literature on the subject of representing the information learned by neural networks in a human understandable format. We developed an optimized implementation of the various algorithms using TensorFlow. Users can just plug in their trained networks and look at the activation maps, deconvolution outputs and even the deepdream representation of their models in TensorBoard. The library quickly gained recognition and has been used by researchers from all over the globe. 
			\end{rSubsection}
			\vspace{0.1 in}
			\begin{rSubsection}{Robust road detection for autonomous vehicles (Masters thesis work)}{Jun, 2014 - May, 2015}{}{}	{}
				\item Studied existing literature on vision-based road detection techniques, implemented the state of the art algorithms and studied their performance. Found certain defects in the existing methods based on the experiments, such as failure under specular reflections from the surface and not being able to detect large white painted portions and proposed an algorithm to overcome them. \\
				Deployed a parallelized version the code on small, low-cost development boards, BeagleBone Black and Raspberry Pi, thus presenting proof of the method being deployable on an actual vehicle and able to function in real time. Findings presented at the IEEE conference, Tencon in Singapore.
			\end{rSubsection}
			\vspace{0.1 in}
			\begin{rSubsection}{Kalman Filter Based GPS tracking  (Internship, ISRO)}{Jan, 2013 - May, 2013}{}{}	{}
			\item There are 2 processes happening in a GPS receiver- acquisition and tracking. Acquisition is done at a coarser frequency and thus it is a more time consuming procedure than tracking. Ionospheric scintillations, which are a common occurrence during evenings cause outage of GPS signal and hence loss of phase lock, thus requiring re-acquisition. I developed a Kalman filter based tracking loop that solves this problem and makes the receiver more robust. Kalman filter is a weighted filter, which ignores the measurements during outage and relies only on the state information, and retains phase lock. 
			\end{rSubsection}
			\vspace{0.1 in}
			
			\begin{rSubsection}{Portfolio Optimizer Design  (Cerebellum Capital/ Infocusp)}{Jan, 2016 - Mar, 2016}{}{}	{}
			\item The trading signals obtained using ML techniques do not necessarily follow all the real life stock trading constraints. These include constraints like limits on short sale, unavailability of certain stocks, the inclusion of nonlinear transaction costs and also hedging against risk. Designed an optimizer that takes as input an unoptimized trading signal and outputs the set of orders that maximize returns, minimize risk and also follow all the trading constraints. Was guided by Prof. Lisa Borland of Stanford University, who also works with CCI.
			\end{rSubsection}
			\vspace{0.1 in}
			
			\begin{rSubsection}{National Data Science Bowl (Kaggle Contest)}{}{}{}	{}
			\item As an intern at Infocusp, participated in the NDSB kaggle contest and learnt different feature engineering techniques and CNN based methods for classification of ocean plankton images. Participated in number of Kaggle contests after this and learnt a variety of data analysis methods from the kernels.
			\end{rSubsection}
			
			\vspace{0.1 in}
		\end{rSection}
		
	\begin{rSection}{Online Courses Accomplished}
		
		\textbf{Machine Learning} \hfill {Stanford University}\\
		\textit{Coursera}
		\vspace{0.1 in}
		\\
		\textbf{Statistical Learning Theory} \hfill{Stanford University}\\
		\textit{Lagunita Stanford}
		\vspace{0.1 in}
		\\
		\textbf{Data Scientist's Toolbox} \hfill {John Hopkins University}\\
		\textit{Coursera}
		\vspace{0.1 in}
		\\
		\textbf{Digital Signal Processing} \hfill{EPFL}\\
		\textit{Coursera}
		\vspace{0.1 in}
		\\
		\textbf{Autonomous Navigation for Flying Robots} \hfill{Technical Universitat De Munich}\\
		\textit{edX}
		\vspace{0.1 in}
		\\
		\textbf{Image And Video Processing} \hfill {Duke University}\\
		\textit{Coursera}
		\vspace{0.1 in}
		\\
	\textbf{Cryptography-I} \hfill {Stanford University}\\
	\textit{Coursera}
	\vspace{0.1 in}	
		\end{rSection}
		
		\begin{rSection}{Positions Of Responsibility}	%------------------------------------------------
		
		\begin{rSubsection}{Placement Coordinator}{Nirma University}{}{}
			\item Facilitating placement proceedings of B.Tech, EC batch of 2013
			\item Responsible for developing contacts with corporate recruitment teams of several firms for placements 
			\item Organized sessions on personality development, group discussions, and mock interviews
		\end{rSubsection}
		
		%------------------------------------------------
		
		\begin{rSubsection}{Organizer, Teach group}{}{}{}
		\item Formed  a group of volunteers to teach some underprivileged kids at a construction site near Nirma University 
		\end{rSubsection}
		
	\end{rSection}
	
\begin{rSection}{Personal Information}
		\begin{tabular}{ @{} >{\bfseries}l @{\hspace{4ex}} l }
			Nationality &  Indian \\
			Languages &  English, Hindi and Gujarati (mother tongue)\\
			Interests & Fitness, Meditation, traveling to mountains and reading fiction
		\end{tabular}
		\end{rSection}
	
\end{document}
